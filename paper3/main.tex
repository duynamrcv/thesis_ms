\chapter{Predictive Reconfiguration Control for Multi-Robot Formation in Cluttered Environments}\label{paper3}

\noindent{\normalsize Submitted in:\\
\textit{IEEE Transactions on Control of Network Systems}\\
Code: {\tt\url{https://github.com/duynamrcv/prc}}\\
Video: {\tt\url{https://youtu.be/oAhgOOhFSCA}}
}
\vspace{1cm}

\noindent\textit{\textbf{Abstract}}

Reconfiguration control is essential for multi-robot systems to adapt their formations in response to environmental conditions when carrying out complex tasks. This chapter introduces a novel approach called predictive reconfiguration control (PRC) to navigate a swarm of robots through cluttered environments with narrow passages such as valleys or tunnels. The robot swarm is modeled as a directed sensing graph where each node represents a robot capable of collecting real-time data on the environment and the states of nearby robots within its communication range. The information from each node is used as input to the controller to adjust its parameters in two control modes, \textit{``formation''} and \textit{``tailgating''}, which serve as the basis for formation adaptation. A set of cost functions is then introduced to represent swarm constraints, including formation shape, reference speed, movement direction, and collision avoidance. These cost functions also allow for the prediction of the swarm's states so that model predictive control solvers can be used to minimize the cost for optimal control signals. Results from a number of comparisons and evaluations show that the proposed controller is not only capable of navigating the robot swarm through challenging environments with narrow passages but also outperforms other state-of-the-art formation control techniques in most performance metrics. Software-in-the-loop tests have been conducted to verify the validity of the proposed controller for practical scenarios.
% \end{abstract}

\noindent\textbf{\textit{Keywords:}}
formation control, multi-robot system, distributed control, reconfiguration control, model predictive control
% \end{keywords}

\section{Introduction}

Formation control of multiple robots enables various real-world applications such as mapping, construction, and search and rescue~\cite{9306908,Oh2015}. These missions are often performed in harsh environments with clustered environments. To effectively cope with those tasks, it is crucial to ensure the safety of the robots both in their surrounding environments and with each other.

Natural formation collectives, such as school of fish or flocking of birds, reveal that coordinated navigation can be achieved through decentralized decision-making~\cite {Nagy2010}. Their motion can be explained by a set of simple rules based on the local information exchange, including repulsion that steers an agent away from its neighbors, cohesion that attracts the agent to the group, migration that orients its motion in a preferred direction, and additional repulsion from obstacles to avoid collisions with the environment~\cite{Reynolds1987}. Various swarm models based on the above rules have been implemented for swarm robots using artificial potential fields (APF)~\cite{736776,Berlinger2021,9565893}. For instance, in~\cite{736776}, formation behaviors are integrated with other navigational behaviors to guide a robotic team to reach navigational goals, avoid collision, and maintain formation shape. Other research in~\cite{Berlinger2021} demonstrates the 3D collective behaviors with a swarm of fish-inspired miniature underwater robots using only implicit communication. However, achieving effective obstacle avoidance in limited space environments using potential field methods remains challenging for real robot swarms, as the density of obstacles can affect the robot's speed when navigating through such environments~\cite{131810}.

In practice, this issue can be addressed by tuning parameters (i.e., preferred speed, cohesion, repulsion, and other coefficients) specific to the environment and swarm configurations~\cite{Elkilany2020,Vsrhelyi2018}. Selecting the appropriate set of parameters is also challenging when the robot formation moves through different environments due to increased model complexity and the higher number of tunable parameters. In~\cite{Elkilany2020}, a swarm robot model is proposed that controls the contraction/expansion of the robot formation using the APF method and an artificial neural network to optimize potential force parameters to adapt to environmental conditions. Experiments performed with three Turtlebot3 robots show that the proposed approach maintains formation distance and can adapt to the movement space, compared to the traditional APF approach. However, it is designed to create a formation structure that cannot navigate through narrow spaces. In~\cite{Vsrhelyi2018}, a communication-aware flocking control for a drone swarm is proposed using an evolutionary optimization framework to select appropriate order parameters and fitness functions to maximize the velocity and cohesion of the swarm. The drone moving in front must always promptly notify the drones behind to avoid crowding into the wall. Results demonstrated with thirty drones moving in tight formation and remaining within a limited area in which the drones avoided collisions with each other and with virtual obstacles by dividing and merging. The control algorithm shows the ability to arrange and expand formations in large spaces but does not demonstrate the ability of formation transitions in narrow environments. Parametric optimization APF-based approaches generally adjust the parameters of behavioral functions, forcing them to alter their formation shape according to environmental influences, but they cannot control the formation shape.

Alternative approaches using optimization for achieving navigation and/or formation control of a swarm have gained considerable attention in the last decade due to their significant impacts on swarm sustainability and their ability to address swarm constraints~\cite{Beaver2021}. The works~\cite{7828016,Wu2020} also demonstrated the remarkable potential of modern optimization-based motion planners for ensuring collision avoidance in multi-robot systems, although these planners are designed for individual point-to-point transitions and do not generate self-organized cohesive flight similar to biological swarms. Recent studies suggest that predictive controllers can improve the safety of robot swarms by predicting and optimizing the agents’ future behavior in an iterative process~\cite{Soria2021,8950150}. Model predictive control computes the control action of a system as the solution to an optimization problem that explicitly accounts for the robot dynamics and actuation constraints. The main limitation of MPC for performing these problems is its high computational cost~\cite{Darby2012,Bui2022}. Fortunately, the development of computational techniques and the support of various powerful libraries can handle this disadvantage in recent years~\cite{Soria2021,Bui2022,2020SciPy-NMeth}. %Moreover, the computation can be shared among all agents according to a distributed MPC (DMPC) formulation. With DMPC, every robot solves an optimal problem locally and then communicates its solution to others to allow global coordination~\cite{9562281,Vargas2022}. 

In ~\cite{Soria2021}, a swarm navigation method using MPC for moving through a clustered environment is proposed and successfully implemented on a real drone system. Inspired by swarm behavior, the algorithm proposes corresponding cost functions and constraints for the robot swarm. The algorithm has proven its ability to navigate, maintain speed, and ensure safety. However, they are being deployed in a centralized manner, which is not feasible in practice moving far from the coordination center. An alternative distributed MPC version~\cite{9562281} is carried out with distributed computing implementation for homogeneous robots, demonstrating the possibility of implementing distributed MPC in a real system. Nonetheless, this approach still ensures that robots maintain a distance in large spaces without considering formation transitions in narrow environments. Maintaining formation while moving through tight environments poses a high risk of collision due to the conflict between preserving the original shape and avoiding obstacles.

These limitations result in the essential of an effective strategy to change the original shape of the formation to another to enhance the rational swarm's motion in narrow spaces. In other studies, strategies based on the structure of narrow environments involve distributed decision-making to enable a new shape of formation~\cite{9565893,AlonsoMora2018}. In~\cite{AlonsoMora2018}, the authors proposed formation change control to adapt to narrow environments with both static and dynamic obstacles using a set of target formation shapes. The approach optimizes parameters such as position, direction, and formation size using a consensus mechanism based on network-wide communication, i.e., a centralized solution based on the limited communication assumption. The formation shapes can be determined by a human designer or calculated automatically. The article employs a distributed consensus mechanism to compute the convex hull of robot positions and determine minimum/maximum positions in the desired movement direction for the entire swarm, subsequently assigning robots to positions within the formation. In this study, however, formation transition was achieved by reassigning robots to virtual points on the reference formation and designing collision-free trajectories to those points, without the robots self-configuring based on local interactions with the environment. Our previous work~\cite{10417519} also changed the shape of the V-shape formation based on the effect of the obstacle to maneuver the robot swarm to a narrow corridor. This V-shaped formation can be shrunk by closing the two wings of the V-shaped formation to the point of forming a straight line depending on the width of the narrow passage. Nevertheless, these approaches do not explicitly consider the physical limitations of the robots, which can result in infeasible control inputs and unexpected collisions in certain settings. Furthermore, the ability of each robot to make its own decisions based on environmental information is still limited.

Motivated by the consistent progress and technological advancements in applications of multi-robot formation in various real-world scenarios, in this paper, we aim to design an optimal deformation control strategy for a decentralized multi-robot team to ensure safety in narrow space environments. The contributions of the paper can be summarized as threefold:
\begin{enumerate}
    \item A perceptual deformation control strategy is proposed to effectively navigate a multi-robot formation moving through narrow environments. Each robot is equipped with local sensors and communication modules to collect information from the surrounding environment and its neighbors for distributed decision-making. The formation thus can be shrunk/expanded or transformed to the line formation according to the environment.
        \item A model prediction-based control strategy is proposed to achieve the navigation requirements of maintaining formation, velocity, and direction, while effectively avoiding collisions. The proposed method models these objectives and constraints into fitness functions that can be easily expanded and deployed to multiple robots simultaneously.
    \item The feasibility and effectiveness of the proposed strategy are demonstrated by simulation and comparison results. A software-in-the-loop experiment was also implemented using flying robots to validate abilities in real-world applications. We also release the source code of the proposed strategy.
\end{enumerate}

The remaining sections of this paper are organized as follows. Section~\ref{sec:problem} describes the formation model. Section~\ref{sec:propose} presents the proposed model prediction-based perceptual deformation control strategy to navigate the multi-robot formation for ensuring collision avoidance in the narrow space environment. Simulation results, comparisons, and a software-in-the-loop experimental validation using aerial robots are given in Section~\ref{sec:result} to highlight the feasibility and efficiency of the proposed strategy. The paper ends with conclusions drawn in Section~\ref{sec:conclusion}.

\section{Formation Background}\label{sec:problem}
Consider a swarm $\mathcal{N}$ of $N$ robots labeled $i\in\left\{1,...,N\right\}$. The swarm is modeled as a directed sensing graph $\mathcal{G}=\left(\mathcal{V},\mathcal{E}\right)$, where vertex set $\mathcal{V} = \left\{1,..., N\right\}$ represents the robots, and edge set $\mathcal{E}\subseteq\mathcal{V}\times \mathcal{V}$ includes robot pairs $\left(i, j\right)\in\mathcal{E}$ for which robot $i$ can sense robot $j$. Denote $\mathcal{N}_i=\left\{j\in\mathcal{V}|\left(i,j\right)\in\mathcal{E}\right\}\subset\mathcal{V}$ as the set of $N_i$ neighbors of a robot $i$ in $\mathcal{G}$.

In this work, the dynamics of the robots are represented in discrete time. Denote $p_i(k),v_i(k),u_i(k)\in\mathbb{R}^3$ respectively be the position, velocity and control input of robot $i$ at time $t(k) = k\tau$, where $\tau$ is the sampling period. The robots in the swarm are homogeneous with a body radius $r$. Each robot is equipped with an inertial measurement unit (IMU) to determine its position and orientation, a range sensor to scan the environment, and a wireless ad-hoc network module to carry out peer-to-peer communication with other robots. In this work, the communication delay between each pair of robots is negligible~\cite{AlonsoMora2018,9527169}. The range sensor provides a $360^\circ$ field of view with the scanning area $S_s$ of radius $r_s$, as shown in Figure~\ref{fig:model}. Its point data obtained at time $t(k)$ is represented by set $\mathcal{M}_i(k)=\left\{m\right\}$.
\begin{figure}
    \centering
    \includegraphics[width=0.48\textwidth]{paper3/images/model.pdf}
    \caption{Illustration of a robot with its range sensor having the scanning area $S_s$ (dashed gray circle) of radius $r_s$ and set $\mathcal{M}_i=\{m\}$ (green) of the acquired point data.}
    \label{fig:model}
\end{figure}

According to~\cite{Soria2021}, the robot in the swarm can be represented as a discrete linear system as follows:
\begin{equation}
    x_i(k+1)=A_ix_i(k) + B_iu_i(k),
\end{equation}
where $A_i$ and $B_i$ are system matrices, $u_i$ is input acceleration, and $x_i=\left[p_i;v_i\right]\in\mathbb{R}^6$ is a state vector including position and velocity. The velocities and accelerations are bounded, i.e., $v_\text{min}\leq v_i(k)\leq v_\text{max}$ and $u_\text{min}\leq u_i(k)\leq u_\text{max}$.
\section{Predictive Reconfiguration Control}\label{sec:propose}

\begin{figure}
    \centering
    \includegraphics[width=0.7\textwidth]{paper3/images/diagram.pdf}
    \caption{Diagram of the proposed predictive reconfiguration control strategy.}
    \label{fig:diagram}
\end{figure}

The aim of reconfiguration control is to drive the robot swarm through a cluttered environment having narrow passages. The swarm adheres to the following constraints: (\textit{C1}) maintain certain desired shapes; (\textit{C2}) move along a prioritized direction $u_\text{ref}\in\mathbb{R}^{3}$; (\textit{C3}) obtain the desired speed $\bar{v}_\text{ref}\in\mathbb{R}$; and (\textit{C4}) ensure no collision with other neighbors or obstacles in the environment. To address this problem, we propose a predictive reconfiguration control system as shown in Figure~\ref{fig:diagram}. Inputs to this system include point cloud data from range sensors and the states of neighbor robots. Depending on the environment structure inferred from the point cloud data, the controller operates in either the \textit{``formation''} or \textit{``tailgating''} mode. The \textit{``formation''} mode maintains the desired shape, whereas the \textit{``tailgating''} mode transforms the formation into a line to navigate through narrow spaces like valleys or tunnels. The controller is designed based on a weighted-sum of five cost functions to meet constraints $(C1)-(C4)$. An optimal solver is then used to generate control signals for low-level controllers.

Let $\delta_{ij}\in\mathbb{R}^3,\forall j\in \mathcal{N}_i$ be the vector representing the desired position of robot $i$ with respect to neighbor $j$. The formation is obtained via the following constraint~\cite{Dong2016,6798711}:
\begin{equation}
    \lim_{k\to\infty}{\left(p_j(k)-p_i(k)+\kappa\delta_{ij}\right)}=0,\quad\forall i,j\in\{1,...,N\}, i\neq j
\end{equation}
where $\kappa\in[0,1]$ is a scaling factor representing the shrinkage level of the formation. The desired relative position of robot $i$ in the formation is then described as:
\begin{equation}
    p^*_i(k)=\dfrac{1}{N_i}\sum_{j\in\mathcal{N}_i}{\left(p_j\left(k\right)+\kappa\delta_{ij}\right)}.
    \label{eqn:formation}
\end{equation}

During the transition between modes, each robot determines a leader as a reference to determine its position. The leader is selected based on the inner product $\tilde{p}_{ij}$ of the difference between robot $j$ in the neighbor set $\mathcal{N}_i$ and robot $i$, $p_j-p_i$, and the desired direction, $u_\text{ref}$, as follows:
\begin{equation}
    \tilde{p}_{ij} = \left\langle (p_j-p_i),u_\text{ref}\right\rangle.
    \label{eqn:tildep}
\end{equation}

A positive value of $\tilde{p}_{ij}$ indicates that robot $j$ is in front of robot $i$ in the $u_\text{ref}$ direction, and vice versa. Let $\mathcal{P}_i$ be the set of inner products for all robots $j$ in the neighbor set $\mathcal{N}_i$, $\mathcal{P}_i=\left\{\tilde{p}_{ij}\right\}$. Leader robot ${l_i}$ of robot $i$ is chosen as the closest robot in front of it, i.e.,
\begin{equation}
     l_i=\begin{cases}
    \arg\min_{j}\left\{\tilde{p}_{ij}\in\mathcal{P}_i\vert\tilde{p}_{ij}\geq0\right\} & \exists~\tilde{p}_{ij}\geq0\\ 
    -1 & \text{otherwise}
     \end{cases}
    \label{eqn:li}
\end{equation}

\begin{algorithm}
\caption{Pseudocode of the leader selection}
\label{alg:ls}
\ForEach{$j\in\mathcal{N}_i$}{
    Compute inner product $\tilde{p}_{ij}$\tcc*[r]{Eq. \ref{eqn:tildep}}
    $\mathcal{P}_i\leftarrow\tilde{p}_{ij}$\;
}
Select leader $l_i$ for robot $i$ to follow\tcc*[r]{Eq. \ref{eqn:li}}
\Return $l_i$\;
\end{algorithm}

Algorithm~\ref{alg:ls} presents the leader selection process. 

In the \textit{``tailgating''} mode, robot~$i$ needs to align and keep a distance~$\bar{d}_\text{ref}\in\mathbb{R}$ with its leader~$l_i$. This can be formulated as follows:
\begin{equation}
    \lim_{k\to\infty}{\left\Vert p_{l_i}(k)-p_i(k)\right\Vert}=\bar{d}_\text{ref}
    \label{eqn:tailcon}
\end{equation}
The desired relative position of robot~$i$ then can be determined as follows:
\begin{equation}
    p_i^*(k)= p_{l_i}(k)-\bar{d}_\text{ref}u_\text{ref}.
    \label{eqn:tailgating}
\end{equation}

Using equations \eqref{eqn:tildep} and \eqref{eqn:tailgating}, the formation problem is converted into tracking the desired position $p_i^*$, which can be handled by predictive controllers. 

\subsection{Predictive control design} 
\begin{figure*}
    \centering
    \includegraphics[width=\textwidth]{paper3/images/perception.pdf}
    \caption{The process of estimating the environment's width from the robot's range sensor.}
    \label{fig:perception}
\end{figure*}
The proposed predictive controller uses the desired position $p_i^*$ as the reference and a set of cost functions to fulfill formation constraints $(C1)-(C4)$. The functions include tracking cost $J_{t,i}$, direction cost $J_{d,i}$, speed cost $J_{s,i}$, obstacle avoidance cost $J_{o,i}$, inter-agent collision cost $J_{i,i}$, and control effort cost $J_{u,i}$. Let $P\in\mathbb{N}^+$ be the prediction horizon, which is finite and shifts forward at each time step; $(\cdot)(k+l|k )$, $l \in\{0,...,P\}$, be the predicted value of $(\cdot)(k+l )$ when the information at time $t(k)$ is available; $X_i(k)\in\mathbb{R}^{6P}$ be the sequence of the predicted states $x_i(k+l|k)$ over the horizon $l\in\{1,...,P\}$; and $U_i(k)\in\mathbb{R}^{3P}$ be the sequence of the predicted control inputs $u_i(k)$ over the horizon $l\in\{0,...,P-1\}$. The predictive reconfiguration control can be modeled as a non-convex optimization problem as follows:
\begin{equation}
\begin{aligned}
    \min_{U_i(k)}&\left(J_{t,i}(k)+J_{s,i}(k)+J_{d,i}(k)+\right.\\
    &\left.J_{o,i}(k)+J_{i,i}(k)+J_{u,i}(k)\right)
\end{aligned}
    \label{eqn:J}
\end{equation}
subject to:
\begin{equation}
    \begin{aligned}
        &x_i(k+l+1|k)=Ax_i(k+l|k)+Bu_i(k+l|k),\\
        &x_i(k|k)=x_i(k),\\
        &v_\text{min}\leq v_i(k+l|k)\leq v_\text{max},\\
        &u_\text{min}\leq u_i(k+l|k)\leq u_\text{max},\\
    \end{aligned}
    \label{eqn:constraints}
\end{equation}
with $l\in\{1,...,P\}$, and $i\in\mathcal{N}$. The cost functions are defined as follows.

\subsubsection{Tracking cost}\label{sec:tracking_term}
The tracking term aims to drive the robots toward their reference positions to achieve the desired formation shape. It is defined as the square error between the desired position~$p_i^*$ and the predicted position~$p_i$ of robot $i$ as follows:
\begin{equation}
    J_{t,i}(k)=w_t\sum_{l=1}^P{\left\Vert p^*_i(k+l|k)-p_i(k+l|k)\right\Vert^2},
\end{equation}
where $w_t$ is a positive tracking weight.

\subsubsection{Speed cost}
The speed cost is used to maintain the desired speed $\bar{v}_\text{ref}$ of the swarm. It is defined as the squared difference between the actual and the desired speed of the robots as follows:
\begin{equation}
    J_{s,i}(k)=w_s\sum_{l=1}^P\left(\left\Vert v_i(k+l|k)\right\Vert^2-\bar{v}_\text{ref}^2\right)^2,
\end{equation}
where $w_s$ is a positive weight.

\subsubsection{Direction cost}
This cost directs the robots to move in the desired direction $u_\text{ref}$. It is computed based on the normalized dot product between velocity $v_i$ and the desired direction $u_\text{ref}$ of robot $i$. It is equal to zero when the velocity perfectly aligns with the reference direction and otherwise increases proportionally with the degree of misalignment. It is given as follows:
\begin{equation}
    J_{d,i}(k)=w_d\sum_{l=1}^P{\left(1-\dfrac{\left\langle v_i\left(k+l|k\right),u_\text{ref}\right\rangle^2}{\left\Vert v_i(k+l|k)\right\Vert^2}\right)^2},
\end{equation}
where $w_d$ is a positive direction weight.

\subsubsection{Collision avoidance cost}
To avoid collisions, the distance from a robot to any obstacle must be greater than the robot's radius $r$ and the distance between any two robots must be greater than $2r$. Let $d_{ij}=\left\Vert p_j-p_i\right\Vert$ be the distance between robots $i$ and $j$, and $d_{im}$ be the distance between robot $i$ and obstacle $m$. The constraints for collision avoidance  are given as follows:
\begin{equation}
\begin{aligned}
    d_{im}(k+l|k)&\geq r \quad i\in\mathcal{N}, m\in\mathcal{M}_i(k)
    \label{eq:obsContraint}
\end{aligned}
\end{equation}
\begin{equation}
\begin{aligned}
    d_{ij}(k+l|k)&\geq 2r \quad i\in\mathcal{N},j\in\mathcal{N}_i
    \label{eq:robotContraint}
\end{aligned}
\end{equation}

In this work, constraint (\ref{eq:obsContraint}) is represented via obstacle avoidance cost $J_{o,i}$ defined as a logistic function as follows~\cite{8202163}:   

\begin{equation}
    J_{o,i}(k) = w_o\sum_{l=1}^P \dfrac{1}{1 + \exp{\left(\alpha\left(d_{im}^\text{min}(k+l|k) - r\right)\right)}},
\end{equation}
where $w_o > 0$ is a constant weight, $\alpha > 0$ is a smoothness parameter, and
\begin{equation}
    d_{im}^\text{min}(k+l|k)=\min\left\{d_{im}(k+l|k)|m\in\mathcal{M}_i\right\}.
\end{equation}

Similarly, constraint (\ref{eq:robotContraint}) is represented via an inter-agent collision cost $J_{i,i}$ defined as follows~\cite{736776}:
\begin{equation}
    J_{i,i}(k)=\dfrac{w_i}{N_i}\sum_{l=1}^P{\sum_{j\in\mathcal{N}_i}}F_{ij}(k+l|k),
\end{equation}
where $w_i>0$ is a constant weight and  
\begin{equation}
    F_{ij}(k+l|k)=\begin{cases}
        0   & \text{if } d_{ij}(k+l|k) \geq \beta r\\
        \dfrac{\beta r-d_{ij}(k+l|k)}{(\beta-2)r}    & \text{if } 2r < d_{ij}(k+l|k) < \beta r\\
        \infty  & \text{if } d_{ij}(k+l|k) \leq 2r
    \end{cases}
\end{equation} 
with $\beta>2$ being the influence ratio of the neighbors.

\subsubsection{Control effort cost}
The control effort cost is used as a penalty term to minimize the control signal. It is defined as:
\begin{equation}
    J_{u,i}(k)=w_u\sum_{l=0}^{P-1}\left\Vert u_i(k+l|k)\right\Vert^2,
\end{equation}
where $w_u>0$ is a constant control weight.

\subsection{Formation reconfiguration}\label{sec:obs_aware}

Depending on the environment's width, the control system can determine the formation shape and scaling factor $\kappa$. The process of estimating the environment's width is illustrated in Figure~\ref{fig:perception}. Robot $i$ first obtains point cloud data $\mathcal{M}_i$ (green) from its local sensor and then selects a point set $\mathcal{M}_{ui}$ (red) in front of the robot along the moving direction $u_\text{ref}$ as follows:
\begin{equation}
    \mathcal{M}_{ui} = \left\{\mathcal{M}_{i}\vert\left\langle\left(p_i-\mathcal{M}_{i}\right),u_\text{ref}\right\rangle<0\right\}.
    \label{eqn:mui}
\end{equation}
The DBSCAN algorithm~\cite{10.5555/3001460.3001507} is then used to divide $\mathcal{M}_{ui}$ into two clusters corresponding to the left (blue) and right (yellow) sides of the robot. Data points $\mathcal{M}_{i,l}$ and $\mathcal{M}_{i,r}$ from those clusters with the shortest distance to $u_\text{ref}$ are then selected. Using these points, the environment's width is computed as:
\begin{equation}
    w_e= \left\Vert\left(\mathcal{M}_{i,r}-\mathcal{M}_{i,l}\right)\times u_\text{ref}\right\Vert
    \label{eqn:we}
\end{equation}
The pseudocode to estimate the environment's width is presented in Algorithm~\ref{alg:we}. 

On the other hand, the formation's width $w_f$ is predefined for each specific formation shape. The scaling factor $\kappa$ then can be computed based on the environment's and formation's width as follows:
\begin{equation}
    \kappa = 
    \begin{cases} 
        \dfrac{w_e - 2r}{w_f} & \text{if } w_e \geq \lambda r \\
        0 & \text{otherwise}
    \end{cases}
    \label{eqn:kappa}
\end{equation}
where $\lambda > 2$ is a scaling coefficient determining the environment's width at which the PRC switches its mode. 

\begin{algorithm}[h!]
\caption{Pseudocode to estimate the environment's width}
\label{alg:we}
Get point set $\mathcal{M}_{ui}$ in front of robot $i$ in moving direction $u_\text{ref}$\tcc*[r]{Eq. \ref{eqn:mui}}
Cluster $\mathcal{M}_{ui}$ for the left and right sides of the robot using DBSCAN\;
Find point pair $\left(\mathcal{M}_{i,l},\mathcal{M}_{i,r}\right)$, whose distance to $u_\text{ref}$ is minimum\;
Compute the environment's width $w_e$\tcc*[r]{Eq. \ref{eqn:we}}
\Return $w_e$\;
\end{algorithm}

\begin{algorithm}[h!]
\caption{Pseudocode of the PRC}
\label{alg:our}
Get data point set $\mathcal{M}_i$\;
\If{$\mathcal{M}_i$ is empty}{
    mode $\leftarrow$ \textit{``formation''}\;
    $\kappa \leftarrow 1.0$\;
}
\Else{
    Get the environment's width $w_e$\tcc*[r]{Alg. \ref{alg:we}}
    \If{$w_e$ is None}{
        mode $\leftarrow$ \textit{``formation''}\;
        $\kappa \leftarrow 1.0$\;
    }
    \Else{
        \If{$w_e\leq\lambda r$}{
            mode $\leftarrow$ \textit{``tailgating''}\;
        }
        \Else{
            mode $\leftarrow$ \textit{``formation''}\;
            Estimate the desired formation width $w_f$\;
            \If{$w_e-2r\leq w_f$}{
                Compute the scaling factor $\kappa$\tcc*[r]{Eq. \ref{eqn:kappa}}
            }
            \Else{
                $\kappa\leftarrow1.0$\;
            }
        }
    }
}
\Switch{mode}{
\Case{``formation''}
{
    Get the desired position $p_i^*$\tcc*[r]{Eq. \ref{eqn:formation}}
}
\Case{``tailgating''}
{
    Select a leader to follow\tcc*[r]{Alg. \ref{alg:ls}}
    Get the desired position $p_i^*$\tcc*[r]{Eq. \ref{eqn:tailgating}}
}
}
Establish the formation cost function\tcc*[r]{Eqs. \ref{eqn:J}-\ref{eqn:constraints}}

Minimize the cost function to obtain the optimal control signal $u_i^*$\tcc*[r]{MPC solver~\cite{2020SciPy-NMeth}}

\Return $u_i^*$\;
\end{algorithm}

Algorithm~\ref{alg:our} presents the pseudocode of the PRC. At each time step, the system estimates the width of the environment to determine the formation mode and the scaling factor $\kappa$. It then computes the desired position $p_i^*$ for each robot and establishes the formation cost function using equations \eqref{eqn:J} - \eqref{eqn:constraints}. This problem is then solved to find the optimal control signal $u_i^*$ using common nonlinear programming (NLP)
solvers, such as the Sequential Least Squares Programming (SLSQP)~\cite{kraft1988software}. In this work, we implemented the solver in Python using the optimization software library SciPy~\cite{2020SciPy-NMeth}.
\section{Results and Discussion}\label{sec:result}

\begin{figure*}
    \centering
    \begin{subfigure}[b]{0.495\textwidth}
    \includegraphics[width=\textwidth]{paper3/images/path_scen1.pdf}
    \caption{Scenario 1 - Motion paths}
    \end{subfigure}
    \begin{subfigure}[b]{0.495\textwidth}
    \includegraphics[width=\textwidth]{paper3/images/path_scen2.pdf}
    \caption{Scenario 2 - Motion paths}
    \end{subfigure}
    \begin{subfigure}[b]{0.495\textwidth}
    \includegraphics[width=\textwidth]{paper3/images/correlation_scen1.pdf}
    \caption{Scenario 1 - Correlation between the number of  robots (bar chart) in each mode and the scale factor (black line) over time}
    \label{fig:cor1}
    \end{subfigure}
    \begin{subfigure}[b]{0.495\textwidth}
    \includegraphics[width=\textwidth]{paper3/images/correlation_scen2.pdf}
    \caption{Scenario 2 - Correlation between the number of  robots (bar chart) in each mode and the scale factor (black line) over time}
    \label{fig:cor2}
    \end{subfigure}
    \caption{Trajectories and formation shapes of the robots controlled by the PRC in two evaluating scenarios.}
    \label{fig:path}
\end{figure*}

A number of simulations, comparisons, and software-in-the-loop tests have been conducted to evaluate the performance of the PRC with details as follows.

\subsection{Evaluation Setup}
The swarm includes five identical robots, each with a radius of $r=0.2$~m, a maximum speed of $\left\Vert \mathbf{v}_\text{max}\right\Vert=1.5$~m/s, and a maximum control input $\left\Vert \mathbf{u}_\text{max}\right\Vert=2.0$~m/s$^2$. The robot is equipped with a range sensor having the sensing range of $r_s=3$~m. The desired formation shape is set to a pentagon, the reference velocity is $\bar{v}_\text{ref}=1$~m/s, and the desired direction is $\mathbf{u}_\text{ref}=[1,0,0]^T$. The environments include two structures, both having narrow passages, as depicted in Figure~\ref{fig:path}.

The metrics used for evaluation include the success rate, mean \textit{order} $\Phi$, mean speed (m/s), mean formation error $\varepsilon$ (m), and acceleration cost $\Gamma$ (m$^2$/s$^4$) \cite{Zhang2021}. The \textit{order} metric~\cite{Vicsek1995} measures the heading consensus of the robots and is computed as:
\begin{equation}
    \Phi=\dfrac{1}{N}\left\Vert\sum_{i=1}^N{\dfrac{\mathbf{v}_i}{\left\Vert \mathbf{v}_i\right\Vert}}\right\Vert
\end{equation}
Its value ranges from 0 to 1, with 1 indicating the robots have the same direction. The \textit{formation error} measures the deviation between the desired and actual positions of the robots and is calculated as~\cite{6798711}:
\begin{equation}
    \varepsilon_i = \left\Vert \mathbf{p}_i-\mathbf{p}^*_i\right\Vert
\end{equation} 
The acceleration cost indicates the control effort and is given by:
\begin{equation}
    \Gamma = \dfrac{1}{n}\sum_{i=1}^N{\left\Vert \mathbf{u_i}(k)\right\Vert^2}
\end{equation} 

The comparing methods include the behavior-based reconfiguration control (BRC) \cite{Vsrhelyi2018} and the predictive formation control (PFC)~\cite{9562281}. In evaluation, each method is run 10 times for each scenario.   

\subsection{Results}
\label{subsec:results}

\begin{figure*}
    \centering
    \begin{subfigure}[b]{0.495\textwidth}
    \includegraphics[width=\textwidth]{paper3/images/velocity_scen1.pdf}
    \caption{Scenario 1 - Speed}
    \label{fig:speed1}
    \end{subfigure}
    \begin{subfigure}[b]{0.495\textwidth}
    \includegraphics[width=\textwidth]{paper3/images/velocity_scen2.pdf}
    \caption{Scenario 2 - Speed}
    \label{fig:speed2}
    \end{subfigure}
    \begin{subfigure}[b]{0.495\textwidth}
    \includegraphics[width=\textwidth]{paper3/images/order_scen1.pdf}
    \caption{Scenario 1 - \textit{Order} $\Phi$}
    \label{fig:order1}
    \end{subfigure}
    \begin{subfigure}[b]{0.495\textwidth}
    \includegraphics[width=\textwidth]{paper3/images/order_scen2.pdf}
    \caption{Scenario 2 - \textit{Order} $\Phi$}
    \label{fig:order2}
    \end{subfigure}
    \begin{subfigure}[b]{0.495\textwidth}
    \includegraphics[width=\textwidth]{paper3/images/error_scen1.pdf}
    \caption{Scenario 1 - \textit{Formation error} $\varepsilon$}
    \label{fig:error1}
    \end{subfigure}
    \begin{subfigure}[b]{0.495\textwidth}
    \includegraphics[width=\textwidth]{paper3/images/error_scen2.pdf}
    \caption{Scenario 2 - \textit{Formation error} $\varepsilon$}
    \label{fig:errorr2}
    \end{subfigure}
    \caption{Comparison results of three control methods in two scenarios.}
    \label{fig:comparison}
\end{figure*}

\begin{figure}
    \centering
    \includegraphics[width=0.8\textwidth]{paper3/images/scalability.pdf}
    \caption{Effect of the swarm size on system performance, including the mean \textit{order} $\Phi$ and formation error $\varepsilon$.}
    \label{fig:scalability}
\end{figure}

Figure \ref{fig:path} shows the formation results for both scenarios. After takeoff, the robots rapidly form the desired pentagon shape, then change their formation to pass through tight passages and finally transform back to the original desired shape. Figures~\ref{fig:cor1}-\ref{fig:cor2} depict the scaling factor $\kappa$ and the number of robots within each control mode during operation. When the environment's width is large, the scaling factor $\kappa$ stays at 1 to maintain the desired shape. As the environment narrows, $\kappa$ gradually decreases leading to more robots switching to \textit{``tailgating''} mode. The formation thus shrinks until $\kappa$ reaches 0, the point at which all robots are in \textit{``tailgating''} mode. Upon exiting the narrow corridor, $\kappa$ increases, allowing the formation to expand back to its original shape. The PRC hence provides reconfiguration capabilities for the robot swarm to adapt to complex environmental conditions.

\begin{table*}
\centering
\caption{Comparison between BRC, PFC, and the proposed PRC}
\label{tbl:analys}
\begin{tabular}{C{0.8cm}C{1.2cm}C{1.8cm}C{1.8cm}C{2.8cm}C{2.2cm}C{2.5cm}}
\hline \hline
Scen.             & Method & Success rate  & Mean \textit{order} $\Phi$ & Mean speed (m/s) ($v_\text{ref}=1$~m/s) & Mean formation error $\varepsilon$ (m) & Acceleration cost $\Gamma$ (m$^2$/s$^4$) \\ \hline
\multirow{3}{*}{1  } & BRC      & \textbf{10/10} & 0.9890     & 1.0249     & 0.3048               & 69.6589    \\
                     & PFC     & 8/10  & \textbf{0.9934}     & 1.0639     & 0.6376               & \textbf{23.7442}    \\
                     & PRC    & \textbf{10/10} & 0.9824     & \textbf{0.9863}     & \textbf{0.2423}               & 25.0894    \\ \hline
\multirow{3}{*}{2}   & BRC      & 6/10  & 0.9883     & \textbf{0.9887}     & 0.6872               & 53.8718    \\
                     & PFC     & 0/10  & \textbf{0.9953}     & 1.0470      & 0.4593               & \textbf{19.0365}    \\
                     & PRC    & \textbf{9/10}  & 0.9830     & 0.9800       & \textbf{0.3217}               & 21.8559   \\ \hline \hline
\end{tabular}
\end{table*}
 
Figure~\ref{fig:comparison} presents comparison results between the PRC and other control methods. In terms of speed, the proposed controller achieves more stable velocities with values closer to the reference $v_\text{ref}$ in both scenarios, as indicated via the average and maximum/minimum values shown in Figures~\ref{fig:speed1}-\ref{fig:speed2}. 
For the \textit{order} metric, the PRC exhibits large fluctuation at start but quickly converges to a high consensus among the robots with the \textit{order} value reaching 1, as shown in Figures~\ref{fig:order1}-\ref{fig:order2}. Both the BRC and PFC also perform well, although the BRC shows more variation than the PRC during the transition phase. In terms of formation maintenance, the PRC outperforms other methods with the smallest average error, as shown in Figures~\ref{fig:error1}-\ref{fig:errorr2}. These results are further confirmed in Table~\ref{tbl:analys}, which presents comparison data. The proposed PRC shows high performance in all metrics, with the highest success rate and the smallest formation error in both scenarios. The PFC fails in scenario 2 due to its limitation in reconfiguring the formation. The BRC has good performance in speed and mean order. It however introduces high acceleration costs, indicating that the method is not energy efficient.       

In another experiment, we vary the swarm size between 4, 5, 7, 10, 15, and 20 robots and measure the mean \textit{order} $\Phi$ and formation error $\varepsilon$ to evaluate the scalability of the proposed method. The result in Figure \ref{fig:scalability} shows that as the number of robots increases, the PRC maintains strong consensus among robots with a mean \textit{order} close to~1. The formation error, excluding the formation generation and the transition stage, remains low with variations around 0.05~m. The PRC is therefore scalable with stable performance across different swarm sizes.

\subsection{Software-in-the-loop verification}
\begin{figure*}
    \centering
    \begin{subfigure}[b]{0.56\textwidth}
    \includegraphics[width=\textwidth]{paper3/images/tunnel.pdf}
    \caption{The cave-like environment}
    \label{fig:gazebo_tunnel}
    \end{subfigure}
    \begin{subfigure}[b]{0.42\textwidth}
    \includegraphics[width=\textwidth]{paper3/images/hummingbird.pdf}
    \caption{The drone model~\cite{Bui2022,Furrer2016}}
    \label{fig:gazebo_hummingbird}
    \end{subfigure}
    \caption{The robot and environment structure used for software-in-the-loop tests.}
    \label{fig:sil}
\end{figure*}
\begin{figure*}
    \centering
    \begin{subfigure}[b]{0.325\textwidth}
    \includegraphics[width=\textwidth]{paper3/images/gazebo_01.png}
    \caption{}
    \end{subfigure}
    \begin{subfigure}[b]{0.325\textwidth}
    \includegraphics[width=\textwidth]{paper3/images/gazebo_02.png}
    \caption{}
    \end{subfigure}
    \begin{subfigure}[b]{0.325\textwidth}
    \includegraphics[width=\textwidth]{paper3/images/gazebo_03.png}
    \caption{}
    \end{subfigure}
    \begin{subfigure}[b]{0.325\textwidth}
    \includegraphics[width=\textwidth]{paper3/images/gazebo_04.png}
    \caption{}
    \end{subfigure}
    \begin{subfigure}[b]{0.325\textwidth}
    \includegraphics[width=\textwidth]{paper3/images/gazebo_05.png}
    \caption{}
    \end{subfigure}
    \begin{subfigure}[b]{0.325\textwidth}
    \includegraphics[width=\textwidth]{paper3/images/gazebo_06.png}
    \caption{}
    \end{subfigure}
    \caption{Reconfiguration process of the robot swarm in a SIL test: (a) initial positions of the robots; (b) form the desired pentagon shape; (c) shrink the formation in adaption to the environment; (d)-(e) switch to \textit{``tailgating''} mode to travel through the narrow passage; (f) transform back to the desired shape.}
    \label{fig:snap}
\end{figure*}

\begin{figure*}
    \centering
    \begin{subfigure}[b]{0.495\textwidth}
    \includegraphics[width=\textwidth]{paper3/images/gazebo_path.pdf}
    \caption{The formation and motion paths}
    \label{fig:gazebo_path}
    \end{subfigure}
    \begin{subfigure}[b]{0.495\textwidth}
    \includegraphics[width=\textwidth]{paper3/images/gazebo_correlation.pdf}
    \caption{Number of robots and scaling factor $\kappa$}
    \label{fig:gazebo_mode}
    \end{subfigure}
    \begin{subfigure}[b]{0.495\textwidth}
    \includegraphics[width=\textwidth]{paper3/images/gazebo_speed.pdf}
    \caption{The speed profile}
    \label{fig:gazebo_speed}
    \end{subfigure}
    \begin{subfigure}[b]{0.495\textwidth}
    \includegraphics[width=\textwidth]{paper3/images/gazebo_order.pdf}
    \caption{The \textit{order} metric $\Phi$}
    \label{fig:gazebo_order}
    \end{subfigure}
    \begin{subfigure}[b]{0.495\textwidth}
    \includegraphics[width=\textwidth]{paper3/images/gazebo_error.pdf}
    \caption{The \textit{formation error} $\varepsilon_i$}
    \label{fig:gazebo_error}
    \end{subfigure}
    \begin{subfigure}[b]{0.495\textwidth}
    \includegraphics[width=\textwidth]{paper3/images/gazebo_computation.pdf}
    \caption{The computational time}
    \label{fig:gazebo_time}
    \end{subfigure}
    \caption{Results of the SIL tests}
    \label{fig:gazebo}
\end{figure*}

We have carried out software-in-the-loop (SIL) tests to evaluate the performance of the proposed controller in practical conditions. The environment is a narrow space that consists of two large obstacles forming a cave-like structure as shown in Figure~\ref{fig:gazebo_tunnel}. The robots include five homogeneous Hummingbird quadrotors\footnote{Source code used to setup SIL tests in Gazebo - {\tt\url{https://github.com/duynamrcv/hummingbird_simulator}}} obtained from the RotorS simulator~\cite{Furrer2016} with an arm length of 0.17~m, a mass of 0.716~kg, the rotor thrust constant of $1.6\time10^{-2}$~N/A, and the rotor drag constant of $8.54858\times10^{-6}$~Nm/A, as depicted in Figure~\ref{fig:gazebo_hummingbird}. Each robot is equipped with a range sensor to collect point cloud data of the environment, a positioning module for localization, and a communication module to interact with other robots.

Figure~\ref{fig:snap} presents the formation reconfiguration process as the swarm navigates through the environment. The robots continuously collect data about the environment and based on it adjust their formation to ensure safe operation. Figure~\ref{fig:gazebo} provides a detailed view of the result. Each UAV determines its mode and desired position based on the perception of the surrounding environment and information about its neighbors. The robots together form the relevant shape in a decentralized manner, as shown in Figures~\ref{fig:gazebo_path} - \ref{fig:gazebo_mode}. Requirements for speed, order, and formation accuracy are also met, as depicted in Figures~\ref{fig:gazebo_speed} - \ref{fig:gazebo_error}. Moreover, the computational time per controller iteration, shown in Figure~\ref{fig:gazebo_time}, indicates that the system can operate at a sampling rate of 10 Hz in the worst-case scenario, which is sufficient for real-time robot operation.

\subsection{Discussion}
Evaluation and comparison results show the key properties of the proposed control method as follows:

\subsubsection{Decentralization} The PRC is fully decentralized as each robot makes decisions based solely on its own sensor data and information from its neighbors. Unlike the approach in~\cite{AlonsoMora2018}, which requires system-wide communication to obtain information on all robots, the PRC utilizes only one-hop communication between the robot and its neighbors to update predictive states.

\subsubsection{Reliability}

The PRC can navigate the robots through complex environments with desirable performance metrics such as low formation error, stable formation direction, and high speed. Unlike previous studies ~\cite{Elkilany2020,Vsrhelyi2018,Soria2021,AlonsoMora2018} which only shrink or expand the formation to adapt to environment variations, the PRC enables the robots to completely transform to a new formation to safely maneuver through tight spaces.

\subsubsection{Scalability}
The PRC can operate with different swarm sizes, for example from 4 to 20 robots, without requiring any modifications to the algorithm. It maintains reliable performance and consistent metric values across various swarm sizes. 

\subsubsection{Robustness}
The PRC provides robustness to operate in different environmental structures. Through real-time data obtained from local sensors and communication networks, the swarm can dynamically expand, shrink, or transform into a line shape to optimize its ability to navigate through narrow spaces.
\section{Conclusion}\label{sec:conclusion}
In this work, we have presented an optimal predictive reconfiguration control method to guide a swarm of robots through cluttered environments with varying path widths. The controller features two modes, \textit{``formation''} and \textit{``tailgating''}, and a scaling factor that enable the swarm to adapt its shape to environmental conditions. A set of cost functions is introduced to enforce formation constraints and safety requirements while enabling state prediction for enhanced control performance. Evaluation results show that the proposed controller effectively navigates the robot swarm through complex environments with narrow passages. The control performance is superior in most evaluation metrics compared to two other state-of-the-art methods. Software-in-the-loop tests further confirm the validity and practicability of the proposed controller.

