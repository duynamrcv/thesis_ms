\section{Introduction}
Formation control of multiple robots is becoming increasingly important due to its capability to perform challenging tasks in complex environments such as disaster response, search and rescue, tracking of multiple targets, and swarm-based reconnaissance~\cite{9306908,Oh2015}. This approach also enhances system resilience as any malfunction within the swarm can be mitigated by reconfiguring the formation. The reconfiguration also enables the swarm to navigate through cluttered environments where a fixed topology is insufficient to adapt the formation to varying conditions. Formation reconfiguration control therefore is essential to ensure the effective operation of a robot swarm.

In formation control, a popular approach is based on natural collectives, such as school of fish or flocking of birds~\cite {Nagy2010}, where movement is driven by a set of simple behaviors: \textit{repulsion} that steers an agent away from its neighbors; \textit{cohesion} that attracts the agent to the group; \textit{migration} that orients its motion in a preferred direction; and \textit{obstacle repulsion} to avoid collisions~\cite{Reynolds1987}. Those behaviors are often used with artificial potential fields (APF) to produce control signals for each robot~\cite{736776,Berlinger2021,9565893, 7434587}. In particular, formation and navigational behaviors are combined in~\cite{736776} as concurrent asynchronous processes to guide a group of robots to reach their goals in a desired formation. In~\cite{Berlinger2021}, the AFP is used to coordinate behaviors among fish-inspired miniature underwater robots to obtain  
collective capabilities without having direct communication. However, behavior-based methods are limited in guiding the swarm through complex environments because a set of fixed behaviors lack the flexibility to handle abrupt changes in the environment~\cite{Zhang2023}.

To address this limitation, several studies propose a hybrid approach that combines the AFP with optimization techniques to better integrate behaviors. For example, the APF is used with an artificial neural network in~\cite{Elkilany2020} to optimize potential force parameters so that the swarm can adapt to environmental conditions. Similarly, an evolutionary optimization framework is introduced in \cite{Vsrhelyi2018} to select parameters and fitness functions that enhance the velocity and cohesion of the swarm. Nevertheless, optimizing parameters is insufficient to address the main problem of the behavior-based approach, which depends on a set of fixed behaviors. 

Recently, an approach using optimal control to manage swarm constraints and predict agents' states has been introduced for reliable formation~\cite{Beaver2021,Soria2021,8950150}. In~\cite{7828016,Wu2020}, optimization-based motion planners are used for individual point-to-point transitions to maintain the formation shape while avoiding collisions in multi-robot systems. In~\cite{Soria2021}, model predictive control (MPC) is introduced for swarm navigation in which cost functions and constraints are used to regulate speed, ensure safety, and guide the drones through the environment. The system however is centralized and dependent on a primary agent. A distributed MPC in~\cite{9562281} offers decentralized computation for formation control, but it is only practical in spacious cluttered environments. Maintaining a specific formation shape in tight spaces is challenging due to conflicts between shape preservation and collision avoidance. 

In another direction, several studies propose control techniques that reconfigure the formation in response to sudden changes in the environment structure \cite{AlonsoMora2018,9013071,9981858,10417519}. In~\cite{AlonsoMora2018}, a distributed method capable of reconfiguring the formation is introduced for environments with both static and dynamic obstacles. The method computes a movable convex region from the robots' current positions and then repositions them within this region based on the desired movement direction. However, the transition relies on reassigning robots to predefined virtual points on the reference formation rather than utilizing information obtained from the environment. In~\cite{9981858}, affine transformations are used to modify formation shapes for obstacle avoidance in dense environments, but they require information to be shared among all the robots. Our previous work~\cite{10417519} adapts a V-shape formation to guide a robot swarm through narrow corridors. The two wings of the V-shape can expand or contract to form a suitable shape based on real-time information about the available space. Nonetheless, this approach has not considered the physical limits of the robots, which could lead to infeasible control inputs and unexpected collisions in certain scenarios. Besides, the ability of each robot to make decisions based on local sensor data remains limited.

Building upon recent progress in multi-robot formation, this work introduces a novel technique named predictive reconfiguration control (PRC) for safe and effective navigation of a decentralized multi-robot swarm in cluttered environments. The robots are equipped with local sensors and communication modules to collect information about the environment and the states of their neighboring robots. The PRC is implemented on each robot to enable adaptive formation capabilities. The contributions of this work are threefold:
\begin{enumerate}
    \item Model the formation as a directed sensing graph where each node represents a robot capable of sensing its surroundings and communicating with its neighbors. This representation allows the system to be decentralized and the formation to be formulated as an optimization problem.
    \item Define a set of cost functions that represent the formation constraints and performance. The functions ensure not only the desired formation shape but also the swarm's performance, including the desired velocity, direction, and obstacle avoidance. The cost functions are designed to be compatible with existing solvers for MPC, thereby simplifying the implementation.
    \item Propose a predictive reconfiguration controller with two modes, \textit{``formation''} and \textit{``tailgating''}, capable of adapting the formation shape in response to environmental changes. Extensive simulations and comparisons have been conducted to evaluate the robustness, scalability, and effectiveness of the proposed controller. Software-in-the-loop tests have also been conducted to verify its practical applicability. The source code of the proposed controller is publicly available for further research and practical implementation.
\end{enumerate}

The remaining sections of this paper are organized as follows. Section~\ref{sec:problem} describes the formation model. Section~\ref{sec:propose} presents the proposed formation reconfiguration control method. Section~\ref{sec:result} shows simulations, comparisons, and software-in-the-loop experimental results. The paper ends with conclusions drawn in Section~\ref{sec:conclusion}.
