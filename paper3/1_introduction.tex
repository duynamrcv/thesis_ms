\section{Introduction}
Formation control of multiple robots is gaining importance due to its capability to perform challenging tasks in complex environments such as disaster response, search and rescue, tracking of multiple targets, and swarm-based reconnaissance ~\cite{9306908,Oh2015}. It also enhances system resilience as the malfunction of a certain robot can be overcome by reconfiguring the swarm formation. Reconfiguration is also essential for a robot swarm to work in complex environments where a fixed topology is insufficient for the robots to navigate through. Formation reconfiguration control therefore plays a key role in the functioning operation of a robot swarm.

In formation control, a popular approach is based on natural formation collectives, such as school of fish or flocking of birds ~\cite {Nagy2010}, in which their motion is conducted by a set of simple behaviors, including repulsion that steers an agent away from its neighbors, cohesion that attracts the agent to the group, migration that orients its motion in a preferred direction, and additional repulsion from obstacles to avoid collisions with the environment~\cite{Reynolds1987}. Those behaviors have been implemented in swarm robots via artificial potential fields (APF)~\cite{736776,Berlinger2021,9565893}. In~\cite{736776}, formation behaviors are integrated with other navigational behaviors to guide a robot group to reach their goals while avoiding collision and maintaining the desired formation shape. In~\cite{Berlinger2021}, 3D collective behaviors are used to control a swarm of fish-inspired miniature underwater robots without having direct communication. The AFP is used to coordinate these behaviors to obtain collective capabilities among the robots. The behavior-based methods, however, are insufficient in adapting to changes in environmental conditions, especially in narrow spaces, since they introduce an abrupt change in the potential fields~\cite{Zhang2023}.

Several improvements that involve a hybrid approach have been proposed to address the adaptation issue. In~\cite{Elkilany2020}, The APF method is combined with an artificial neural network to optimize potential force parameters for better adaptation of the robot swarm to environmental conditions. In~\cite{Vsrhelyi2018}, an evolutionary optimization framework is introduced to select appropriate parameters and fitness functions to maximize the velocity and cohesion of the swarm. While this parametric optimization APF-based approach generally adjusts the parameters of behavioral functions to alter their formation shape according to environmental conditions, formation transitions remain inefficient for narrow environments. The changes in narrow environments create conflicts between maintaining formation and avoiding collision behaviors, making parameter optimization insufficient to completely address the issue.

An approach based on optimal control has been introduced recently for swarm formation with the capability of handling swarm constraints~\cite{Beaver2021}. In~\cite{7828016,Wu2020}, optimization-based motion planners are employed with the introduction of individual point-to-point transition constraints to ensure collision avoidance in multi-robot systems. In~\cite{Soria2021}, a swarm navigation method using model predictive control (MPC) capable of moving through a cluttered environment is proposed for a real drone system. Inspired by swarm behavior, the algorithm proposes corresponding cost functions and constraints for the robot swarm. The algorithm has proven its ability to navigate, maintain speed, and ensure safety. However, they are being deployed in a centralized manner, which is not feasible in practice moving far from the coordination center. An alternative distributed MPC version~\cite{9562281} is carried out with distributed computing implementation for homogeneous robots, demonstrating the possibility of implementing distributed MPC in a real system. Nonetheless, this approach still ensures that robots maintain a distance in large spaces without considering formation transitions in narrow environments. Maintaining formation while moving through tight environments poses a high risk of collision due to the conflict between preserving the original shape and avoiding obstacles. Model predictive control (MPC) can improve the safety of robot swarms by predicting and optimizing the agents' future behavior through an iterative process~\cite{Soria2021,8950150}. Its main limitation is the high computational cost~\cite{9910373,Bui2022}. However, the development of computational techniques and software libraries can handle this disadvantage in recent years~\cite{Soria2021,Bui2022,2020SciPy-NMeth}. 

These limitations result in the essential of an effective strategy to change the original shape of the formation to another to enhance the rational swarm's motion in narrow spaces. In other studies, strategies based on the structure of narrow environments involve distributed decision-making to enable a new shape of formation~\cite{9565893,AlonsoMora2018}. In~\cite{AlonsoMora2018}, the authors proposed formation change control to adapt to narrow environments with both static and dynamic obstacles using a set of target formation shapes. The approach optimizes parameters such as position, direction, and formation size using a consensus mechanism based on network-wide communication, i.e., a centralized solution based on the limited communication assumption. The formation shapes can be determined by a human designer or calculated automatically. The article employs a distributed consensus mechanism to compute the convex hull of robot positions and determine minimum/maximum positions in the desired movement direction for the entire swarm, subsequently assigning robots to positions within the formation. In this study, however, formation transition was achieved by reassigning robots to virtual points on the reference formation and designing collision-free trajectories to those points, without the robots self-configuring based on local interactions with the environment. Our previous work~\cite{10417519} also changed the shape of the V-shape formation based on the effect of the obstacle to maneuver the robot swarm to a narrow corridor. This V-shaped formation can be shrunk by closing the two wings of the V-shaped formation to the point of forming a straight line depending on the width of the narrow passage. Nevertheless, these approaches do not explicitly consider the physical limitations of the robots, which can result in infeasible control inputs and unexpected collisions in certain settings. Furthermore, the ability of each robot to make its own decisions based on environmental information is still limited.

Motivated by the consistent progress and technological advancements in applications of multi-robot formation in various real-world scenarios, in this paper, we aim to design an optimal formation reconfiguration control strategy for a decentralized multi-robot team to ensure safety in narrow space environments. The contributions of the paper can be summarized as threefold:
\begin{enumerate}
    \item A perceptual formation reconfiguration control strategy is proposed to effectively navigate a multi-robot formation moving through narrow environments. Each robot is equipped with local sensors and communication modules to collect information from the surrounding environment and its neighbors for distributed decision-making. The formation thus can be shrunk/expanded or transformed to the line formation according to the environment.
        \item A model prediction-based control strategy is proposed to achieve the navigation requirements of maintaining formation, velocity, and direction, while effectively avoiding collisions. The proposed method models these objectives and constraints into fitness functions that can be easily expanded and deployed to multiple robots simultaneously.
    \item The feasibility and effectiveness of the proposed strategy are demonstrated by simulation and comparison results. A software-in-the-loop experiment was also implemented using flying robots to validate abilities in real-world applications. We also release the source code of the proposed strategy.
\end{enumerate}

The remaining sections of this paper are organized as follows. Section~\ref{sec:problem} describes the formation model. Section~\ref{sec:propose} presents the proposed model prediction-based perceptual formation reconfiguration control strategy to navigate the multi-robot formation for ensuring collision avoidance in the narrow space environment. Simulation results, comparisons, and a software-in-the-loop experimental validation using aerial robots are given in Section~\ref{sec:result} to highlight the feasibility and efficiency of the proposed strategy. The paper ends with conclusions drawn in Section~\ref{sec:conclusion}.
