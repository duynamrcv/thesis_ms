\chapter{Conclusion and Future Works}\label{conclusion}

Driven by the potential impact of multiple UAV systems on numerous missions, such as search and rescue (SAR), this master thesis has presented a set of contributions addressing the autonomous navigation of UAV formations through a confined environment, especially in narrow spaces. A brief summary of the conducted research, the key insights, and contributions are given as follows.

Aiming at the safe navigation of a V-shape formation, \textit{Method 1} addresses formation control by designing a distributed architecture for self-reconfiguration. The proposed method is constructed by several behaviors, that allow a V-shape formation to shrink/expand, which depends on the width of the environment. In a collision-free environment, the method ensures that the shape maintenance and inter-collision-freeness between each pair of UAVs are complete. 

Similarly to our previous work in \textit{Method 1}, \textit{Method 2} also address multi-robot navigation. However, instead of focusing on V-shape formation, the study expands the ability to use various types of configuration. This work presents a reconfiguration control method, named event-based reconfiguration control, which is constructed by several behaviors based on artificial potential fields. The stability of the method is also proven via Lyapunov theory. Thanks to the proposed method, robot formation can navigate safely through a narrow space by changing the scale, or transform to a straight line configuration.

Nevertheless, constraints and limitations of robot systems are not considered in the previous works. As a result, \textit{Method 3} model the reconfiguration control as an optimal problem which considers the limitation of robot systems. The behaviors presented in \textit{Method 2} are converted to objective functions, and the control signal is obtained by minimizing the weighted-sum cost function, subject to constraints. By collecting the data from the environment, the surrounding space is considered to estimate the width of the space. The optimal solution not only ensures the effectiveness of the reconfiguration control but also enhances the safety and collision-free.

In future works, we aim to develop safety-critical reconfiguration control strategies that enhance safe movement and minimize the potential of collision with other robots and environments. Due to narrow spaces, the potential of collision between robots with neighbors or environments becomes considerable. As a result, an improvement related to safety-critical is essential to ensure collision-free. Following this, the non-communication formation reconfiguration control methods can be considered to enhance the autonomy and robustness of the systems. A vision-based perception becomes an effective alternative approach to deal with the mentioned problem.