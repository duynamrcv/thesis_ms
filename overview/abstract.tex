\chapter*{Abstract}
\addcontentsline{toc}{chapter}{Abstract}
% \fontsize{13}{15}\selectfont

Unmanned Aerial Vehicles (UAVs), commonly referred to as drones, have the potential to significantly influence various applications, including inspection, post-disaster assessment, and Search and Rescue (SAR). This is due to their exceptional agility, allowing for free movement in 3D space, and their progressively decreasing cost. To automate the aforementioned tasks using UAVs, researchers have concentrated on enhancing the capability of these vehicles to autonomously navigate unfamiliar environments, utilizing onboard sensors for pose estimation, mapping, and path planning. Advanced methods for deploying multiple robots, including formation as a popular coordination strategy, have been proposed to increase the efficiency of robotic missions, which is particularly crucial in time-sensitive applications like rescue operations. However, coordinating multiple UAVs in a formation within a constrained environment presents several challenges, including maintaining formation and avoiding collisions. Motivated by these challenges, this thesis addresses the formation reconfiguration control problem of a multi-robot formation to safely navigate through narrow environments.

Aiming to the widest formation configuration, i.e. V-shape formation, and the typical assumptions related to communication and navigation, the first approach proposed a distributed self-reconfiguration control strategy. The objective is the safe navigation of a V-shape formation through a narrow space. However, although proposing a complete solution, this approach is still limited to only V-shape configuration. Addressing this limitation, in a follow-up approach, an event-based formation reconfiguration control for multiple formation configurations is proposed. This method provides an effective strategy to transform the formation shape in confined spaces, by collecting data from local sensors equipped on each robot. Moreover, the strategy is demonstrated as stable via Lyapunov theory. Additionally, this type of control mission includes minimizing energy, as well as formation errors during the movement, subject to the constraints and limitations of the robot system. Therefore, an optimal control strategy is developed in the third approach, which transforms the ideal given in the previous work to the optimal solution that meets further requirements in system constraints, as well as enhances the smoothness of the movement.

With the focus on multi-robot coordination and perception-aware active planning for UAVs, the approaches and systems presented in this thesis contribute towards autonomous aerial navigation deployable in confined space scenarios. Furthermore, it is demonstrated that the use of the proposed methods is extremely beneficial for multi-robot control purposes during flights. This leads to more robust methods, contributing to the way towards more safe autonomous navigation of robotic agents.